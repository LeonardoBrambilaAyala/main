\documentclass[letterpaper,12pt]{article}
\usepackage{titlesec}
\titleformat*{\section}{\bfseries}
\usepackage[margin=1in]{geometry}
\usepackage{fontsetup}
\usepackage{polyglossia}
\setmainlanguage[variant=mexican]{spanish}

\begin{document}

\begin{center}
  \textbf{Tarea 5.}\par
  \bigskip
  Leonardo Brambila Ayala.\par
  \bigskip
  Lenguajes de programación\par
  Lic. en Ciencias de la Computación\par
  Universidad de Sonora\par
  Semestre 2023-1\par
\end{center}

\section{Epigramas sobre programación}
\subsection {Top 10-Epigramas}

\subsection* {Me gustaron: }

\begin{enumerate}
\item Es más fácil escribir un programa incorrecto que entender uno correcto.\par
\textbf{$¿$Que significa para mi$?$}\par
\textbf{R =} \text Es mas facil en equivocarse muchas veces en elgo, que entender el como hacerlo de una manera correcta, aunque de manera
incorrrecta se aprende el como hacerlo de una manera correcta.\par
\textbf{$¿$Porque lo elegi$?$}\par
\textbf{R =} \text Porque tanto literalmente y como significa para mi desde mi pubto de vista, se me hace totalmente cierto.

\item Si un oyente asiente con la cabeza cuando le explicas tu programa, despiértalo.\par
\textbf{$¿$Que significa para mi$?$}\par
\textbf{R =} \text Hay que explicar y desarrollar de una manera el habla, para que una persona no se duerma, con la explicacion del codigo, u otra cosa
en la vida cotidiana.\par
\textbf{$¿$Porque lo elegi$?$}\par
\textbf{R =} \text Porque es verdad, porque, no conozco ninguna persona que no lo hiciera, y tambien es aalgo ormal en la vida, que una prsona
empieza a cabeciar o bostezar, cuando le muestran algo o leexpliquen algo, que ya es normal.\par

\item  No vale la pena conocer un lenguaje que no afecta tu forma de pensar sobre la programación.\par
\textbf{$¿$Que significa para mi$?$}\par
\textbf{R =} \text Que no hagamos algo que no vale la pena, si no nos hace crecer de una forma, para asi ser mejor en muchas mas cosas.\par
\textbf{$¿$Porque lo elegi$?$}\par
\textbf{R =} \text Por experiencia, de una manera literal, yo estaba aferrado en c++, sin saber que era una clase, bien, que me explicaran,
eso me hizo cambiar de una manera mas estructurado las cosas.

\item La simplicidad no precede a la complejidad, sino que la sigue.
\textbf{$¿$Que significa para mi$?$}\par
\textbf{R =} \text Una mejor manera de hacer las cosas, es partiendo de la simplicidad, y solamente hacer suscomplicaciones cuando estas, se pueden
mantener bajor control y oportunas.
\textbf{$¿$Porque lo elegi$?$}\par
\textbf{R =} \text Porque yo me complico la vida, y luego cuando la quiero hacer simple, ya es muy tarde, o ya no se siente tan bien.

\item Los programadores no deben ser medidos por su ingenio y su lógica sino por la integridad de su análisis de casos.\par
\textbf{$¿$Que significa para mi$?$}\par
\textbf{R =} \text Pensar sobre una cosa y llevarlo, a un pensamiento, bien estructurado, que uno podria entender, mejor dicho tener un buen enfoque sobre los problemas a complenderlos a fondo, antes de hacer las cosas.\par
\textbf{$¿$Porque lo elegi$?$}\par
\textbf{R =} \text Por que es lo opuesto a lo que hago en muchos casos, tanto en la programacion y en los casos de la vida cotidiana.

\item La computadora más importante es la que hace estragos en nuestros cráneos y siempre busca ese emulador externo satisfactorio. La estandarización de los ordenadores reales sería un desastre y, por lo tanto, probablemente no sucederá.\par
\textbf{$¿$Que significa para mi$?$}\par
\textbf{R =} \text No importa como lo hagamos, nunca sera exactamente igual del como pensamos las cosas, cuando se manifiestan en un estado 
fisico, o de una salida dentro de una pantalla, no sera igual del como o imaginamos o pensamos.\par
\textbf{$¿$Porque lo elegi$?$}\par
\textbf{R =} \text Porque busque en todos lados, el como expresar, mis ideas y pensamientos en un emulador, y no e encontrado, uno, que se aprezca
lo suficiente de como pienso las cosas.

\item Es más fácil cambiar la especificación para adaptarla al programa que viceversa.\par
\textbf{$¿$Que significa para mi$?$}\par
\textbf{R =} \text Pura verdad, es mas facil hacer tu las cosas intentando que cumplan con lo que te dicen y no cumplirlas e itentar que el entorno lo acepte que hacerlo bien, mucho mas facil\par
\textbf{$¿$Porque lo elegi$?$}\par
\textbf{R =} \text Porque es lo que deseo que pase, muchas veces, de una manera literal

\newpage
\item En programación, como en todo, equivocarse es renacer.\par
\textbf{$¿$Que significa para mi$?$}\par
\textbf{R =} \text Aprende de tus errores, es mas facil entender y "evolucionar" de los errores, que de los extotos, te mejoras a ti mismo\par
\textbf{$¿$Porque lo elegi$?$}\par
\textbf{R =} \text Porque es fue lo que me paso, en mis clase de programacion, porque para hacer un programa, que, por ejemplo vimos, por mas de una semana, hice tantas versiones y con tantos errores, que aprendi de alli, y mi mente se expandio.

\item En informática, las invariantes son efímeras.\par
\textbf{$¿$Que significa para mi$?$}\par
\textbf{R =} \text Las coondicciones cambian, alrededor del tiempo, haciendo que sean efimeras, que duran poco tiempo las cosas\par
\textbf{$¿$Porque lo elegi$?$}\par
\textbf{R =} \text Porque estuvehaciendo un programa, que estaba cambeando las condicciones y propiedades, y me gusto.

\item Cuando alguien diga "Quiero un lenguaje de programación en el que sólo tenga que decir lo que deseo que se haga", dale una paleta.\par
\textbf{$¿$Que significa para mi$?$}\par
\textbf{R =} \text Todo el mundo, a tenido, un pensamiento tan ingenuo, que lo comprendes en algun punto, pero es simple un sueño\par
\textbf{$¿$Porque lo elegi$?$}\par
\textbf{R =} \text Porque e tenido esa trase de pensamientos
\end{enumerate}

\newpage

\subsection* {No me gustaron: }

\begin{enumerate}
\item Entre en la rutina temprano: haga el mismo proceso de la misma manera. Acumular modismos. Estandarizar. La única diferencia (!) entre Shakespeare y usted era el tamaño de su lista de modismos, no el tamaño de su vocabulario.\par
\textbf{$¿$Que significa para mi$?$}\par
\textbf{R =} \text La importancia de una adaptacion de idiomas en diferentes ambitos, para mejorrar la creatividad, en los lenguajes que conocemos y conoceremos.
\textbf{$¿$Porque lo elegi$?$}\par
\textbf{R =} \text Por que es verdad, ya que uno necesita practica y creativiadad, que solamente tener un conjunto de conocimientos, y por alguna razon me disgusto esto.

\item La optimización obstaculiza la evolución.\par
\textbf{$¿$Que significa para mi$?$}\par
\textbf{R =} \text El estancamiento de algo, al intentar mejoralo hazta cierto punto, dejara de innovar, ciertas tecnologias, y dejaran de adaptar nuevas ideas, y se quedaran en su zona de confort
\textbf{$¿$Porque lo elegi$?$}\par
\textbf{R =} \text Porque estoy de acuerdo hazta cierto punto, pr que al momento de optimizar ago, o al querer hazerlo, uno muchas veces debe de ininovar cosas y salir de uan zona de confort, pero muchos se quedan, con la idea deoptimizar con o que saben y empreden nueva ideas.

\item Quizás si escribiéramos programas desde pequeños, de adultos podríamos leerlos.\par
\textbf{$¿$Que significa para mi$?$}\par
\textbf{R =} \text Aprender desde pequeños, podria hacernos desarrollar nuestras habilidades, hasta quenos presenten atencion.
\textbf{$¿$Porque lo elegi$?$}\par
\textbf{R =} \text Porque de niños, el objetivo de aprender, es hacerlo, y sobre la marcha comprenderlo, no para que los entiendan.

\item Algunos lenguajes de programación logran absorber el cambio, pero resisten el progreso.\par
\textbf{$¿$Que significa para mi$?$}\par
\textbf{R =} \text Que los cambios de uno o de varias cosas, significan progreso, si un tipo de cambio superficial.
\textbf{$¿$Porque lo elegi$?$}\par
\textbf{R =} \text No tiene sentido , para mi, porque al absorver el cambi, el el progreso mismo

\newpage
\item Cuando entendamos los sistemas basados ​​en el conocimiento, será como antes, excepto que nuestras yemas de los dedos se habrán chamuscado.\par
\textbf{$¿$Que significa para mi$?$}\par
\textbf{R =} \text Que auque entendamos algo, va sdr de la misma manera, que uno no cambiara las cosas simplmente entendiendolas, y al hcer eso "nuestras yemas de los dedos e habran chamuscado", es el desgaste y el costo a ese simple conocimiento.
\textbf{$¿$Porque lo elegi$?$}\par
\textbf{R =} \text Porque, no seria cmo antes en si, al comprender las cosas, solo serian lgeramente mejor, por eso lo elegi, estoy ligeremente en desacuerdo

\item Los tontos ignoran la complejidad. Los pragmáticos lo sufren. Algunos pueden evitarlo. Los genios lo eliminan.\par
\textbf{$¿$Que significa para mi$?$}\par
\textbf{R =} \text Que diferentes personas pueden afrontar las cosas de maneras diferentes, y obteniendo resultados muy similares al final, a veces. 
\textbf{$¿$Porque lo elegi$?$}\par
\textbf{R =} \text Por que es verdad, y por eso no me gusta, ya que sufro, con la complejidad y eso quita tiempo y energias

\item En la búsqueda de lo inalcanzable, la sencillez sólo se interpone en el camino.\par
\textbf{$¿$Que significa para mi$?$}\par
\textbf{R =} \text La sencillez no es la solucion, para muchas cosas, en la vida de una persona
\textbf{$¿$Porque lo elegi$?$}\par
\textbf{R =} \text Porque, por desgracia es verdad, el camino facil, muchas veces no es la solucion y muchas veces lo complica mas

\item Pensemos en toda la energía psíquica gastada en buscar una distinción fundamental entre "algoritmo" y "programa".\par
\textbf{$¿$Que significa para mi$?$}\par
\textbf{R =} \text Cualquier cosa abtracta se puede interpretar de tantas maneras, que hay choqueses en esos pensamientos, haciendo que digan que son diferentes a esto a a quello.
\textbf{$¿$Porque lo elegi$?$}\par
\textbf{R =} \text Porque, no lo comprendo por completo, ya que las dos cosas son abstractas de sde mi punto de vista, que en algun punto sea la misma

\item Siempre que dos programadores se reúnen para criticar sus programas, ambos guardan silencio.\par
\textbf{$¿$Que significa para mi$?$}\par
\textbf{R =} \text Uno no puede esperar criticas, ideas y opiniones, de otras personas, tan facilmente, haciendo que nos retrasemos, o nos confiemos en algo
\textbf{$¿$Porque lo elegi$?$}\par
\textbf{R =} \text Por que me a pasado de una menera literal, y no me gusto eso.

\item En informática, convertir lo obvio en útil es una definición viva de la palabra "frustración".\par
\textbf{$¿$Que significa para mi$?$}\par
\textbf{R =} \text Poner en practica la teoria, es mucho mas dificil que simplmente decirlo como si fuera facil
\textbf{$¿$Porque lo elegi$?$}\par
\textbf{R =} \text Porque es falso en algunos casos, porque es mas facil hacerlo, que literalmente escribir como hacelo y el orque hacerlo asi.
\end{enumerate}

\end{document}
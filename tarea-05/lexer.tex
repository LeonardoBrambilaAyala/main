\documentclass[letterpaper,12pt]{article}
\usepackage{titlesec}
\titleformat*{\section}{\bfseries}
\usepackage[margin=1in]{geometry}
\usepackage{fontsetup}
\usepackage{polyglossia}
\setmainlanguage[variant=mexican]{spanish}

\begin{document}


\begin{center}
  \textbf{Tarea 5.}\par
  \bigskip
  Leonardo Brambila Ayala.\par
  \bigskip
  Lenguajes de programación\par
  Lic. en Ciencias de la Computación\par
  Universidad de Sonora\par
  Semestre 2023-1\par
\end{center}

\section {Descripcion de las las funciones.}

\text (define (read-char* input)\par
\text  (define ch (read-char input))\par
\text  (when debug?\par
\text    (printf "read char ~s from input~%" ch))
\text  ch)
\par
\textbf {R =} 
\text Lee un caracter del imput, mediante la funcion "read-char", y lo que se leyo se guardara en la variable local "ch". Despues verifica en debug y luego imprime un mensaje usando printf.\par

\section* {Seg.Fun}
\text(define (peek-char* input)\par
 \text (define ch (peek-char input))\par
  \text(when debug?
    (printf "peeked char ~s from input~%" ch))
  ch)\par

\textbf {R=}
\text La funcion peek-char*, se utiliza para obtener el proximo aracter de la fuente de entrada (input) sin consumirlo, y este caracter se guardara en el variable ch.\par

\section* {Ter.Fun}
\text(define (char-digit? ch)
 \text (char<=? \# \backslash0 ch \# \backslash9))\par
\textbf {R =}
\text ch: Es el caracter que se va a verificar.\par
\text La función utiliza la función char<=? para comparar el carácter ch con los caracteres 0-9. Devuelve verdadero si esta dentro del rango establecido si no devolvera falso.\par
\text Detrminina si el caracyer pasado como argumento es un digito decimal del 0-9. Su resultado de "char<=?" sera verdadero si el caracter es un dijito, si no sera falso\par

\section* {Cuar.Fun}
\text(define (char-varletter? ch)
  \text(member ch '(\#\backslash x \#\backslash y \#\backslash z) char=?))
\par
\textbf {R =} 
\text ch: Es el caracter que se va a verificar.\par
\text Verifica si el caracter a verificar esta dentro de la lista prmitida, "x, y, z. " La funcion devuelve el primer elemento de la lista cuando es igual a la variable "ch".\par
\text En conclusion verifica si el caracter "ch: es una de las letras dentro del rango, si es asi devolvera verdadero, si no devolvera falso.\par

\section* {Quin.Fun}
\text(define (char-delimiter? ch)
  \text(or (eof-object? ch)
     \text (char-whitespace? ch)
     \text (member ch '(\#\backslash( \#\backslash)))))
\par
\textbf {R =} 
\text ch: Es el caracter que se va a verificar.\par
\text Determina si es caracter, es un delimitador, esto puede inclui espacios en blancos y parevtesis, si es ese caso devolvera verdadero si no falso.\par

\section* {Sex.Fun}
(define (token-open-paren)
  (token 'open-paren \#f lex-line lex-col))\par
Crea una instanci en Toen con su propio tipo open-paren, el valor de falso y las posiciones de las columnas y sus lineas actuales\par

\section* {Sep.Fun}
(define (token-close-paren)
  (token 'close-paren \#f lex-line lex-col))\par
Crea una instanci en Toen con su propio tipo clsose-paren, el valor de falso y las posiciones de las columnas y sus lineas actuales\par

\section* {Opt.Fun}
(define (token-binop type)
  (token 'binop type lex-line lex-col))\par
Toma un argumento y crea una nueva instancia de Token con el tipo binop, el valor del argumento y las posiciones de las columnas y lieas actuales\par

\section* {Novena.Fun}
(define (token-number value col)
  (token 'number value lex-line col))\par

Toma dos argumentos, value y col, y crea una instancia de token con el tipo 'number, el valor value, y las posiciones de las columnas y lineas actuales.\par

 \section* {Decima.Fun}
(define (token-number/+ token)\par
  (token 'number\par
         (token-value token)\par
         (token-line token)\par
         (sub1 (token-col token))))\par

Toma un token como argumento.
Crea un nuevo token , el valor del token original (token-value token), la misma línea, pero con la columna decrementada en 1.
La operación  esta relacionada con la adición (+), ya que la columna se decrementa en 1.
\section* {Onceava.Fun}
(define (token-number/- token)\par
  (token 'number
         (- (token-value token))
         (token-line token)\par
         (sub1 (token-col token))))\par

Crea un nuevo token con el tipo 'number, con el valor negativo del token, en la misma línea (token-line token), pero con la columna decrementada en 1.

\section {Errores y o bugs del codigo}

\begin{enumerate}
\item En la funcion "(define(read-alphanumeric)..)"
\item En la funcion "(define (token-number/- token)...)"
\end{enumerate}

\end{document}